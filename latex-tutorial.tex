\documentclass[blue]{qslides}

% include additional packages
\usepackage{pdfpages}
\usepackage{listings}

% beamer template settings
\setbeamertemplate{caption}[numbered]
\setbeamertemplate{enumerate items}[default]
\setbeamertemplate{bibliography item}[text]

\usepackage{ragged2e}

% listings settings
\lstset{
  basicstyle=\small,
  commentstyle=\color{gray},
  keywordstyle=,
  breaklines=true,
  tabsize=2,
  frame=leftline
}

% hyperref settings
\hypersetup{
  colorlinks=true,
  linkcolor=,
  filecolor=,
  urlcolor=blue
}

% title content
\title{A \LaTeX~Tutorial}
\author[\insertframenumber~of~\inserttotalframenumber]{Johann von Tiesenhausen}
\institute{Ingenuity Labs Research Institute \\ Queen's University, Kingston, Canada}
\date{\today}
\logo{\includegraphics[width=0.5in]{figures/ingenuitylabs_logo.pdf}}

\begin{document}

\frame{\titlepage}

\section{Introduction}

\begin{frame}{Goals of this presentation}

\vspace{1.5em}

\begin{itemize}
  \item \textbf{Not} comprehensive
  \item Starting reference to show that \LaTeX~isn't \textit{that} scary
  \item Provide some tips \& tricks
  \item Tons of useful \LaTeX~tutorials by \href{https://www.overleaf.com/learn}{Overleaf}
\end{itemize}

\vspace{2em}

\begin{center}
  \fbox{\includegraphics[width=0.8\textwidth]{figures/cool.pdf}}
\end{center}

\end{frame}

\begin{frame}{What is \LaTeX?}

\begin{itemize}
  \item From \href{https://www.britannica.com/technology/TeX}{Encycolpedia Britannica}:
  \begin{justify}
    ``TeX, a page-description computer programming language developed during 1977--86 by Donald Knuth, a Stanford University professor, to improve the quality of mathematical notation in his books. Text formatting systems [\ldots] embed plain text formatting commands in a document, which are then interpreted by the language processor to produce a formatted document for display or printing.''
  \end{justify}
  \item\LaTeX is the corresponding software package
  \item TeX consists of the greek letters $\tau$, $\epsilon$, $\chi$, and is pronounced ``lay-tech''
\end{itemize}

\end{frame}

\section{Documents}

\begin{frame}[fragile]{A \LaTeX~document}

\begin{lstlisting}[language=TeX]
\documentclass[12pt]{article}

% remaining preamble goes here

\begin{document}

% content goes here

\end{document}
\end{lstlisting}

\end{frame}

\begin{frame}[fragile]{Creating a title}

\begin{lstlisting}[language=TeX]
\documentclass[12pt]{article}

\title{Automatic Material Classification}
\author{Unal Artan \thanks{Thank you to Natalie \& Johann}}
\date{August 24, 2021}

\begin{document}

\maketitle
...
\end{lstlisting}

\end{frame}

% include resulting pdf page
{
  \setbeamercolor{background canvas}{bg=}
  \includepdf[fitpaper]{figures/title.pdf}
}

\begin{frame}[fragile]{Common commands}

\begin{center}
  \begin{tabular}{ l | l }
    comments              & \verb_% ..._                              \\
    \textbf{bold}         & \verb_\textbf{...}_                       \\
    \textit{italic}       & \verb_\textit{...}_ or \verb_\emph{...}_  \\
    \underline{underline} & \verb_\underline{...}_                    \\
    inline equations      & \verb_$...$_                              \\
    block equations       & \verb_$$...$$_ or \verb_\[...\]_          \\
    \ldots and many more! & \verb_\ldots_                             \\
  \end{tabular}
\end{center}

\vspace{1em}

\begin{center}
  \verb_\include{...}_ is used to insert \LaTeX code from another file in-place
\end{center}

\end{frame}

\begin{frame}[fragile]{Layout commands/info}

\begin{center}
  \begin{tabular}{ l l }
    \hline
    \textbf{Command} & \textbf{Description} \\
    \hline
    \verb_\vspace{...}_ & add vertical space \\
    \verb_\hspace{...}_ & add horizontal space \\
    \hline
    \textbf{Dimension} & \textbf{Description} \\
    \hline
    \verb_pt_ & point, smallest unit of measure \\
    \verb_in_ & inch (72.27 \verb_pt_) \\
    \verb_cm_ & centimeter \\
    \verb_mm_ & millimeter \\
    \verb_em_ & relative to current point size (e.g., for 11\verb_pt_ font, 1\verb_em_ = 11\verb_pt_) \\
    \verb_en_ & half the width of \verb_em_ \\
  \end{tabular}
\end{center}

\end{frame}

\begin{frame}{Typesetting notes}

\begin{itemize}
  \item Extra spaces between words are ignored
  \item An empty line starts a new \textbf{paragraph}
  \item Two backslashes (\textbackslash\textbackslash) \textbf{forces} a line break, but does not start a new paragraph (i.e., no indent)
  \item Periods are treated as the \textbf{end of a sentence}, unless followed by a comma or backslash (e.g., i.e.\textbackslash)
  \item Tilde (\textasciitilde) inserts \textbf{non-breaking whitespace}
  \item \textbf{Opening quotes} are denoted by 1--2 grave accents (` or ``)
  \item \textbf{Closing quotes} are denoted by 1--2 apostrophes (' or '')
\end{itemize}

\end{frame}

\section{Lists}

\begin{frame}[fragile]{Lists I: Itemize}

\begin{columns}
\column{0.5\textwidth}
\begin{lstlisting}[language=TeX]
\begin{itemize}
  \item Lima
  \item[-] Navy
  \item Kidney
  \begin{itemize}
    \item[yes] Bean
    \item[no] Stone
  \end{itemize}
\end{itemize}
\end{lstlisting}

\column{0.5\textwidth}
\begin{itemize}
  \item Lima
  \item[-] Navy
  \item Kidney
  \begin{itemize}
    \item[yes] Bean
    \item[no] Stone
  \end{itemize}
\end{itemize}

\end{columns}
  
\end{frame}

\begin{frame}[fragile]{Lists II: Enumerate}

\begin{columns}
\column{0.5\textwidth}
\begin{lstlisting}[language=TeX]
\begin{enumerate}
  \item One
  \item Two
  \item Three
  \begin{enumerate}
    \item Three Eh
    \item Three Bee
  \end{enumerate}
\end{enumerate}
\end{lstlisting}

\column{0.5\textwidth}
% use alphabet for enumii (must come after `\begin{document}')
\renewcommand{\theenumii}{\alph{enumii}}
\begin{enumerate}
  \item One
  \item Two
  \item Three
  \begin{enumerate}
    \item Three Eh
    \item Three Bee
  \end{enumerate}
\end{enumerate}

\end{columns}

\end{frame}

\section{Tables}

\begin{frame}[fragile]{Tables}

\begin{columns}
\column{0.6\textwidth}
\begin{lstlisting}[language=TeX]
\begin{tabular}{ | r | c c | }
  \hline
       & col1 & col2 \\ 
  \hline
  row1 & r1c1 &      \\
  row2 &      & r2c2 \\
  \hline
\end{tabular}
\end{lstlisting}

\column{0.4\textwidth}
\begin{tabular}{ | r | c c | }
  \hline
        & col1 & col2 \\ 
  \hline
  row1 & r1c1 &      \\
  row2 &      & r2c2 \\
  \hline
\end{tabular}
\end{columns}

\end{frame}

\section{Math}

\begin{frame}[fragile]{Common math syntax}

\begin{center}
  \begin{tabular}{ l l l l }
    \hline Description   & Code                       & Output                \\
    \hline subscript     & \verb-x_y-                 & $x_y$                 \\
    \hline superscript   & \verb-x^y-                 & $x^y$                 \\
    \hline grouping      & \verb-x^{y+z}-             & $x^{y+z}$             \\
    \hline fraction      & \verb-\frac{x}{y}-         & $\frac{x}{y}$         \\
    \hline square root   & \verb-\sqrt{x+y}-          & $\sqrt{x+y}$          \\
    \hline greek letters & \verb-\alpha \beta \gamma- & $\alpha~\beta~\gamma$ \\
    \hline spacing       & \verb-\; \: \, \!-         & contextual            \\
  \end{tabular}
\end{center}

\end{frame}

\begin{frame}[fragile]{Equations}

\begin{lstlisting}[language=TeX]
\begin{equation}
  \beta(s) = \int^\infty_{-\infty} CWT(s, \tau) \; d\tau
  \label{eq:CWTint}
\end{equation}
\end{lstlisting}

\vspace{1em}

\begin{equation}
  \beta(s) = \int^\infty_{-\infty} CWT(s, \tau) \; d\tau
  \label{eq:CWTint}
\end{equation}

\end{frame}

\begin{frame}[fragile]{Multiline equations}

\begin{lstlisting}[language=TeX]
\begin{equation}
  \begin{split}
    a_{1, X} = & a_{1,x} \cos{\alpha} \\
                & - a_{1,z} \sin{\alpha}
  \end{split}
\end{equation}
\end{lstlisting}

\vspace{0em}

\begin{equation}
  \begin{split}
    a_{1, X} = & a_{1,x} \cos{\alpha} \\
                & - a_{1,z} \sin{\alpha}
  \end{split}
\end{equation}

\end{frame}

\section{Figures}

\begin{frame}[fragile,allowframebreaks]{Creating a figure}

\begin{lstlisting}[language=TeX]
\begin{figure}[t]
  \centering
  \includegraphics[height=0.65\textheight]{figures/loader_diagram.png}
  \caption{The Kubota R520s robotic 1-tonne-capacity wheel loader that was used for field experiments.}
  \label{fig:loader}
\end{figure}
\end{lstlisting}

\framebreak

\begin{figure}[t]
  \centering
  \includegraphics[height=0.65\textheight]{figures/loader_diagram.png}
  \caption{The Kubota R520s robotic 1-tonne-capacity wheel loader that was used for field experiments.}
  \label{fig:loader}
\end{figure}

\end{frame}    

\section{References}

\begin{frame}[fragile]{Label references}

\begin{lstlisting}[language=TeX]
``\ldots the Kubota Loader in Figure~\ref{fig:loader}''
\end{lstlisting}

``\ldots the Kubota Loader in Figure~\ref{fig:loader}''

\vspace{2em}

\begin{lstlisting}[language=TeX]
``see Equation~\ref{eq:CWTint}''
\end{lstlisting}

``see Equation~\ref{eq:CWTint}''

\end{frame}

\begin{frame}[fragile]{Bibliography references}

BibTeX entry (.bib files):

\lstinputlisting[basicstyle=\tiny]{references.bib}

\begin{lstlisting}[language=TeX]
``\ldots due to breakthrough research \cite{artan2021}''
\end{lstlisting}

\vspace{0.5em}

``\ldots due to breakthrough research \cite{artan2021}''

\end{frame}

\begin{frame}[fragile]{Inserting a bibliography}

% references (build: xelatex, bibtex, xelatex)

Often requires you to precompile your document, run bibtex, then compile it again with resolved references \ldots

\vspace{0.5em}

\begin{lstlisting}[language=TeX]
\bibliographystyle{ieeetr}
\bibliography{references.bib}
\end{lstlisting}

\vspace{1em}

\bibliographystyle{ieeetr}
\bibliography{references.bib}

\end{frame}

\begin{frame}[fragile]{Hyperlinks}

\begin{lstlisting}[language=TeX]
\href{https://ieeexplore.ieee.org/document/9517696}{Paper}
\end{lstlisting}

\href{https://ieeexplore.ieee.org/document/9517696}{Paper}

\vspace{1.5em}

\begin{lstlisting}[language=TeX]
\hypertarget{link:resolved}{This part}
\hyperlink{link:resolved}{That part}
\end{lstlisting}

\hypertarget{link:resolved}{This part}

\hyperlink{link:resolved}{That part}

\end{frame}

\section{Commands}

\begin{frame}[fragile]{Custom commands}

\begin{lstlisting}[language=TeX]
\newcommand{cmd}[args][default]{def}
\end{lstlisting}

\begin{tabular}{ l | l }
  \verb_cmd_ & name of the command \\
  \verb_args_ & number of parameters \\
  \verb_default_ & default value for optional first parameter \verb_#1_ \\
  \verb_def_ & command body \\
\end{tabular}

\vspace{1.5em}

\begin{lstlisting}[language=TeX]
\newcommand{\proot}[2][]{{^{#1}}\!\!\!\sqrt{#2}}
\[ \proot[3]{x + y} + \proot{x} \]
\end{lstlisting}

\newcommand{\proot}[2][]{{^{#1}}\!\!\!\sqrt{#2}}
\[ \proot[3]{x + y} + \proot{x} \]

\end{frame}

\begin{frame}[fragile,allowframebreaks]{Custom environments}

\begin{lstlisting}[language=TeX]
\newenvironment{name}[args][default]{begdef}{enddef}
\end{lstlisting}

\begin{tabular}{ l | l }
  \verb_name_ & name of the environment \\
  \verb_args_ & number of parameters \\
  \verb_default_ & default value for optional first parameter \verb_#1_ \\
  \verb_begdef_ & \verb_\begin_ command body \\
  \verb_enddef_ & \verb_\end_ command body \\
\end{tabular}

\framebreak

\begin{lstlisting}[language=TeX]
\newenvironment{LARGEcenter}
  {\begin{center}\LARGE}
  {\end{center}}

\begin{tinycenter}
  Thank you for your time!
\end{tinycenter}
\end{lstlisting}

\newenvironment{LARGEcenter}
  {\begin{center}\LARGE}
  {\end{center}}

\begin{LARGEcenter}
  Any questions?
\end{LARGEcenter}

\end{frame}

\end{document}