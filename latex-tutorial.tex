\documentclass[blue]{qslides}

% include additional packages
\usepackage{pdfpages}
\usepackage{listings}
\usepackage{ragged2e}
\usepackage[linesnumbered,ruled]{algorithm2e}

% beamer template settings
\setbeamertemplate{caption}[numbered]
\setbeamertemplate{enumerate items}[default]
\setbeamertemplate{bibliography item}[text]
\setbeamerfont{bibliography item}{size=\scriptsize}
\setbeamercovered{invisible}

% listings settings
\lstset{
  basicstyle=\scriptsize,
  commentstyle=\color{gray},
  keywordstyle=,
  breaklines=true,
  tabsize=2,
  frame=leftline
}

% hyperref settings
\hypersetup{
  colorlinks=true,
  linkcolor=,
  filecolor=,
  urlcolor=blue
}

% custom item color for descriptions
\let\olddescription\description
\let\oldenddescription\enddescription
\renewenvironment{description}{%
    \olddescription
    \let\olditem\item
    \renewcommand{\item}[1][]{\olditem[\color{gray}##1]}
  }{%
    \let\item\olditem
    \oldenddescription
  }

% math operators
\DeclareMathOperator{\CWT}{CWT}

% title content
\title{A \LaTeX~Tutorial}
\author[\insertframenumber~of~\inserttotalframenumber]{Johann von Tiesenhausen}
\institute{Ingenuity Labs Research Institute \\ Queen's University, Kingston, Canada}
\date{June 9\textsuperscript{th}, 2022}
\logo{\includegraphics[width=0.5in]{figures/ingenuitylabs_logo.pdf}}

\begin{document}

\frame{\titlepage}

\section{Introduction}

\begin{frame}{Goals of this presentation}

\vspace{1.5em}

\begin{itemize}
  \item \textbf{Not} comprehensive
  \item Starting reference to show that \LaTeX~isn't \textit{that} scary
  \item Provide some tips \& tricks
  \item Tons of useful \LaTeX~tutorials by \href{https://www.overleaf.com/learn}{Overleaf}
\end{itemize}

\vspace{2em}

\begin{center}
  \fbox{\includegraphics[width=0.8\textwidth]{figures/cool.pdf}}
\end{center}

\end{frame}

\begin{frame}{What is \LaTeX?}

\begin{itemize}
  \item From \href{https://www.britannica.com/technology/TeX}{Encycolpedia Britannica}:

  \vspace{0.5em}
  \begin{minipage}{0.85\textwidth}
    \begin{justify}
      \small
      ``TeX, a page-description computer programming language developed during 1977--86 by Donald Knuth, a Stanford University professor, to improve the quality of mathematical notation in his books.

      Text formatting systems, unlike WYSIWYG (“What You See Is What You Get”) word processors, embed plain text formatting commands in a document, which are then interpreted by the language processor to produce a formatted document for display or printing. TeX marks italic text, for example, as {\it this is italicized}, which is then displayed as this is italicized.''
    \end{justify}
  \end{minipage}
  \vspace{0.5em}
  \item\LaTeX is the corresponding software package
  \item TeX consists of the greek letters $\tau$, $\epsilon$, $\chi$, and is pronounced ``lay-tech''
\end{itemize}

\end{frame}

\section{Documents}

\begin{frame}[fragile]{A \LaTeX~document}

\begin{lstlisting}[language=TeX]
\documentclass[12pt]{article}

% remaining preamble goes here

\begin{document}

% content goes here

\end{document}
\end{lstlisting}

\end{frame}

\begin{frame}[fragile]{Creating a title}

\begin{lstlisting}[language=TeX]
\documentclass[12pt]{article}

\title{Automatic Material Classification}
\author{Unal Artan \thanks{Thank you to Natalie \& Johann}}
\date{August 24, 2021}

\begin{document}

\maketitle
...
\end{lstlisting}

\end{frame}

\begin{frame}[fragile]{Adding sections}

\begin{lstlisting}[language=TeX]
...
\tableofcontents

\section{Wavelet Analysis}

Researchers began studying wavelets in the 30s, because of their inherent cuteness in comparison to waves. 

\subsection{Digression}

Would that this were a whiteboard instead \ldots

\end{document}
\end{lstlisting}

\textbf{NOTE}: Requires two compilations to output the table of contents.

\end{frame}

% include resulting pdf page
{
  \setbeamercolor{background canvas}{bg=}
  \includepdf[fitpaper]{figures/sample/sample.pdf}
}

\begin{frame}[fragile]{Commonly used syntax}

\begin{center}
  \begin{tabular}{ r l }
    \verb_\usepackage{...}_ & import a \LaTeX~package in preamble \\
    \verb_\input{...}_ & input \LaTeX~code from another file in-place \\
    \verb_\include{...}_ & insert \LaTeX~code from another file on separate pages \\
  \end{tabular}
\end{center}

\begin{center}
  \begin{tabular}{ l | l }
    comments              & \verb_% ..._                              \\
    \textbf{bold}         & \verb_\textbf{...}_                       \\
    \textit{italic}       & \verb_\textit{...}_ or \verb_\emph{...}_  \\
    \underline{underline} & \verb_\underline{...}_                    \\
    inline equations      & \verb_$...$_                              \\
    block equations       & \verb_$$...$$_ or \verb_\[...\]_          \\
    ``quotes''            & \verb_`...'_ or \verb_``...''_            \\
    \ldots and many more! & \verb_\ldots_                             \\
  \end{tabular}
\end{center}

\end{frame}

\begin{frame}[fragile]{Dimensions and layout}

\begin{center}
  \begin{tabular}{ l l }
    \hline
    \textbf{Dimension} & \textbf{Description} \\
    \hline
    \verb_pt_ & point, smallest unit of measure \\
    \verb_in_ & inch (72.27 \verb_pt_) \\
    \verb_cm_ & centimeter \\
    \verb_mm_ & millimeter \\
    \verb_em_ & relative to current point size (e.g., for 11\verb_pt_ font, 1\verb_em_ = 11\verb_pt_) \\
    \verb_en_ & half the width of \verb_em_ \\
    \hline
    \textbf{Command} & \textbf{Description} \\
    \hline
    \verb_\vspace{...}_ & add vertical space \\
    \verb_\hspace{...}_ & add horizontal space \\
  \end{tabular}
\end{center}

\end{frame}

\begin{frame}{Typesetting notes}

\begin{itemize}
  \item Extra spaces between words are ignored
  \item An empty line starts a new \textbf{paragraph}
  \item Two backslashes (\textbackslash\textbackslash) \textbf{forces} a line break, but does not start a new paragraph (i.e., no indent)
  \item Periods with trailing whitespace are treated as \textbf{end of sentence}, which can be escaped by a trailing backslash (e.g., i.e.\textbackslash)
  \item Tilde (\textasciitilde) inserts \textbf{non-breaking whitespace}
  \item Adding an asterisk (*) after some environment names will hide their numbering (e.g., section*, figure*, equation*)
  \item Curly braces (\{\ldots\}) may be used as blocks for formatting
\end{itemize}

\end{frame}

\section{Lists}

\begin{frame}[fragile]{Lists}

Packages:

\begin{description}
  \item[enumitem] custom enumerations/nesting
\end{description}

Commands:

\begin{description}
  \item[itemize] bullet points
  \item[enumerate] numbered lists
  \item[description] description lists (used here)
\end{description}

\end{frame}

\begin{frame}[fragile]{Itemize}

\begin{columns}
\column{0.5\textwidth}
\begin{lstlisting}[language=TeX]
\begin{itemize}
  \item Lima
  \item Navy
  \item Kidney
  \begin{itemize}
    \item[yes] Bean
    \item[no] Stone
  \end{itemize}
\end{itemize}
\end{lstlisting}

\column{0.5\textwidth}
\begin{itemize}
  \item Lima
  \item Navy
  \item Kidney
  \begin{itemize}
    \item[yes] Bean
    \item[no] Stone
  \end{itemize}
\end{itemize}
\end{columns}
  
\end{frame}

\begin{frame}[fragile]{Enumerate}

\begin{columns}
\column{0.5\textwidth}
\begin{lstlisting}[language=TeX]
\begin{enumerate}
  \item One
  \item Two
  \item Three
  \begin{enumerate}
    \item Three Eh
    \item Three Bee
  \end{enumerate}
\end{enumerate}
\end{lstlisting}

\column{0.5\textwidth}
% use alphabet for enumii (must come after `\begin{document}')
\renewcommand{\theenumii}{\alph{enumii}}
\begin{enumerate}
  \item One
  \item Two
  \item Three
  \begin{enumerate}
    \item Three Eh
    \item Three Bee
  \end{enumerate}
\end{enumerate}
\end{columns}

\end{frame}

\section{Tables}

\begin{frame}[fragile]{Tables}

Packages:

\begin{description}
  \item[array] tables with fixed-width cells
  \item[tabularx] tables with fixed page width
  \item[multirow] merge rows/columns
\end{description}

Commands:

\begin{description}
  \item[tabular] table environment
\end{description}

\end{frame}

\begin{frame}[fragile]{Tabular}

\begin{columns}
\column{0.5\textwidth}
\begin{lstlisting}[language=TeX]
\begin{tabular}{ | r | l | l | }
  \hline
  Signal & Description & Range \\ 
  \hline
  $\theta_l$ & lift & [0, 604] mm \\
  \hline
  $\theta_d$ & dump & [0, 396] mm \\
  \hline
\end{tabular}
\end{lstlisting}

\column{0.5\textwidth}
{
  \small
  \begin{tabular}{ | r | l | l | }
    \hline
    Signal & Description & Range \\ 
    \hline
    $\theta_l$ & lift cylinder & [0, 604] mm \\
    \hline
    $\theta_d$ & dump cylinder & [0, 396] mm \\
    \hline
  \end{tabular}
}
\end{columns}

\end{frame}

\section{Math}

\begin{frame}[fragile]{Math}

Packages:

\begin{description}
  \item[amsmath] core math functionality
  \item[amssymb] extended mathematical symbols set 
  \item[cases] piecewise notation
  \item[algorithm2e] algorithm environment
\end{description}

Commands:

\begin{description}
  \item[equation] equation environment
  \item[split] multiline equation environment
  \item[align] multiple aligned equations environment
  \item[(pbv)matrix] matrices (similar syntax to tables) 
\end{description}

\end{frame}

\begin{frame}[fragile]{Common equation syntax}

\begin{center}
  \begin{tabular}{ l l l l }
    \hline Description   & Code                       & Output                \\
    \hline subscript     & \verb-x_y-                 & $x_y$                 \\
    \hline superscript   & \verb-x^y-                 & $x^y$                 \\
    \hline grouping      & \verb-x^{y+z}-             & $x^{y+z}$             \\
    \hline fraction      & \verb-\frac{x}{y}-         & $\frac{x}{y}$         \\
    \hline square root   & \verb-\sqrt{x+y}-          & $\sqrt{x+y}$          \\
    \hline greek letters & \verb-\alpha \beta \gamma- & $\alpha~\beta~\gamma$ \\
    \hline spacing       & \verb-\; \: \, \!-         & contextual            \\
  \end{tabular}
\end{center}

\end{frame}

\begin{frame}[fragile]{Equations}

\begin{lstlisting}[language=TeX]
\begin{equation}
  \beta(s) = \int^\infty_{-\infty} CWT(s, \tau) \; d\tau
  \label{eq:CWTint}
\end{equation}
\end{lstlisting}

\vspace{1em}

\begin{equation}
  \beta(s) = \int^\infty_{-\infty} CWT(s, \tau) \; d\tau
  \label{eq:CWTint}
\end{equation}

\end{frame}

\begin{frame}[fragile]{Custom operators and text}

In the preamble:

\begin{lstlisting}[language=TeX]
\DeclareMathOperator{\CWT}{CWT}
\end{lstlisting}

\vspace{1em}

used in \eqref{eq:CWTint} gives:

\begin{lstlisting}[language=TeX]
\[ \beta(s) = \int^\infty_{-\infty} \CWT(s, \tau) \; \textrm{d}\tau \]
\end{lstlisting}

\vspace{1em}

\[ \beta(s) = \int^\infty_{-\infty} \CWT(s, \tau) \; \textrm{d}\tau \]

\end{frame}

\begin{frame}[fragile]{Multiline equations}

\begin{lstlisting}[language=TeX]
\begin{equation}
  \begin{split}
    a_{1, X} & = a_{1,x} \cos{\alpha} - a_{1,z} \sin{\alpha} \\
              & = 42
  \end{split}
\end{equation}
\end{lstlisting}

\begin{equation}
  \begin{split}
    a_{1, X} & = a_{1,x} \cos{\alpha} - a_{1,z} \sin{\alpha} \\
             & = 42
  \end{split}
\end{equation}

\end{frame}

\begin{frame}[fragile]{Matrices}

\begin{columns}
\column{0.5\textwidth}
\begin{lstlisting}[language=TeX]
\[
  \begin{bmatrix}
    0 & -1 & 0 & 0 \\
    1 & 0  & 0 & 0 \\
    0 & 0  & 1 & 0 \\
    0 & 0  & 0 & 1
  \end{bmatrix}
\]
\end{lstlisting}

\column{0.5\textwidth}
\[
  \begin{bmatrix}
    0 & -1 & 0 & 0 \\
    1 & 0  & 0 & 0 \\
    0 & 0  & 1 & 0 \\
    0 & 0  & 0 & 1
  \end{bmatrix}
\]
\end{columns}

% \pause

\only<2->\transdissolve
\uncover<2->{\centering Exactly!}

\end{frame}

\begin{frame}[fragile,allowframebreaks]{Algorithms}

\begin{lstlisting}[language=TeX]
\begin{algorithm}[H]
  \caption{Dynamic Time Warping}
  \KwIn{Time series $X[1..n]$ and $Y[1..m]$}
  \KwOut{Cost matrix $DTW[0..n][0..m]$}
  $DTW[0][0] \gets 0$ \tcp*{warping path root}
  ...

  \For{$i \gets 1$ \textbf{to} $n$}
  {
    ...
  }

  \KwRet{$DTW[1..n][1..m]$}
  \label{alg:DTW}
\end{algorithm}
\end{lstlisting}

\framebreak

\begin{algorithm}[H]
  \DontPrintSemicolon
  \SetAlgoLined
  \scriptsize
  \caption{Dynamic Time Warping}
  \KwIn{Time series $X[1..n]$ and $Y[1..m]$}
  \KwOut{Cost matrix $DTW[0..n][0..m]$}
  $DTW[0][0] \gets 0$ \tcp*{warping path root}
  $DTW[1..n][0] \gets infinity$
  $DTW[0][1..m] \gets infinity$
  \For{$i \gets 1$ \textbf{to} $n$}
  {
    \For{$j \gets 1$ \textbf{to} $m$}
    {
      $cost \gets \abs{X[i] - Y[j]}$ \tcp*{Euclidian distance}
      $DTW[i][j] \gets cost + min$\parbox[t]{.3\linewidth}{(%
        $DTW[i-1, j],$\\
        $DTW[i, j-1],$\\
        $DTW[i-1, j-1])$}
    }
  }
  \KwRet{$DTW[1..n][1..m]$}
  \label{alg:DTW}
\end{algorithm}

\end{frame}

\section{Figures and references}

\begin{frame}{Figures and references}

Packages:

\begin{description}
  \item[graphicx] including graphics
  \item[biblatex] bibliography management
  \item[hyperref] hyperlinks 
\end{description}

Commands:

\begin{description}
  \item[includegraphics] include graphics \ldots
  \item[addbibresource] add bibliography (.bib) file
  \item[printbibliography] insert bibliography
  \item[cite] in-text citation
  \item[href] create hyperlink
\end{description}

\end{frame}

\begin{frame}[fragile,allowframebreaks]{Creating a figure}

\begin{lstlisting}[language=TeX]
\begin{figure}[t]
  \centering
  \includegraphics[height=0.65\textheight]{%
    figures/loader_diagram.png%
  }
  \caption{The Kubota R520s robotic 1-tonne-capacity wheel loader that was used for field experiments.}
  \label{fig:loader}
\end{figure}
\end{lstlisting}

\framebreak

\begin{figure}[t]
  \centering
  \includegraphics[height=0.7\textheight]{%
    figures/loader_diagram.png%
  }
  \caption{The Kubota R520s robotic 1-tonne-capacity wheel loader that was used for field experiments.}
  \label{fig:loader}
\end{figure}

\end{frame}    

\begin{frame}[fragile]{Label references}

\begin{lstlisting}[language=TeX]
``\ldots the Kubota Loader in Figure~\ref{fig:loader}''
\end{lstlisting}

{\small``\ldots the Kubota Loader in Figure~\ref{fig:loader}''}

\vspace{2em}

\begin{lstlisting}[language=TeX]
``used in \eqref{eq:CWTint} gives''
\end{lstlisting}

{\small``used in \eqref{eq:CWTint} gives''}

\vspace{2em}

\textbf{Note}: ``fig:'' and ``eg:'' are not necessary, but they help when writing. ``ch:'' and ``sec:'' are often used for chapters and sections.

\end{frame}

\begin{frame}[fragile]{Bibliography references}

BibTeX entry (.bib files):

\lstinputlisting[basicstyle=\tiny]{references.bib}

Corresponding in-text citation:

\begin{lstlisting}[language=TeX]
``\ldots due to breakthrough research \cite{artan2021}''
\end{lstlisting}

\vspace{0.5em}

{\small``\ldots due to breakthrough research \cite{artan2021}''}

\end{frame}

\begin{frame}[fragile]{Inserting a bibliography}

% references (build: xelatex, bibtex, xelatex)

\textbf{NOTE}: Often requires you to compile your document, run bibtex, then compile it again with resolved references \ldots

\vspace{1em}

\begin{lstlisting}[language=TeX]
\bibliographystyle{ieeetr}
\bibliography{references.bib}
\end{lstlisting}

\vspace{1em}

\bibliographystyle{ieeetr}
\bibliography{references.bib}

\end{frame}

\begin{frame}[fragile]{Hyperlinks}

URL links:

\begin{lstlisting}[language=TeX]
\href{https://ieeexplore.ieee.org/document/9517696}{Paper}
\end{lstlisting}

{\small\href{https://ieeexplore.ieee.org/document/9517696}{Paper}}

\vspace{1.5em}

Internal links:

\begin{lstlisting}[language=TeX]
\hypertarget{link:thisPart}{This part}
\hyperlink{link:thisPart}{That part}
\end{lstlisting}

{\small\hypertarget{link:thisPart}{This part}}

{\small\hyperlink{link:thisPart}{That part}}

\end{frame}

\section{Commands}

\begin{frame}[fragile]{Custom commands}

\begin{lstlisting}[language=TeX]
\newcommand{cmd}[args][default]{def}
\end{lstlisting}

\begin{tabular}{ l | l }
  \verb_cmd_ & name of the command \\
  \verb_args_ & number of parameters \\
  \verb_default_ & default value for optional first parameter \verb_#1_ \\
  \verb_def_ & command body \\
\end{tabular}

\vspace{1.5em}

\begin{lstlisting}[language=TeX]
\newcommand{\proot}[2][]{\; ^{#1} \!\!\! \sqrt{#2}}
\[ \proot[3]{x + y} + \proot{x} \]
\end{lstlisting}

\newcommand{\proot}[2][]{\; ^{#1} \!\!\! \sqrt{#2}}
\[ \proot[3]{x + y} + \proot{x} \]

\end{frame}

\begin{frame}[fragile,allowframebreaks]{Custom environments}

\begin{lstlisting}[language=TeX]
\newenvironment{name}[args][default]{begdef}{enddef}
\end{lstlisting}

\begin{tabular}{ l | l }
  \verb_name_ & name of the environment \\
  \verb_args_ & number of parameters \\
  \verb_default_ & default value for optional first parameter \verb_#1_ \\
  \verb_begdef_ & \verb_\begin_ command body \\
  \verb_enddef_ & \verb_\end_ command body \\
\end{tabular}

\framebreak

\begin{lstlisting}[language=TeX]
\newenvironment{LARGEcenter}
  {\begin{center}\LARGE}
  {\end{center}}

\begin{LARGEcenter}
  Thank you for your time!
\end{LARGEcenter}
\end{lstlisting}

\newenvironment{LARGEcenter}
  {\begin{center}\LARGE}
  {\end{center}}

\begin{LARGEcenter}
  Any questions?
\end{LARGEcenter}

\end{frame}

\end{document}